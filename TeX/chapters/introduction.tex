\section{Introduction}

%% adapt introduction to make it more friendly, but not too verbose.
%%
%% write the outline more specific towards the main idea of the thesis,
%% namely fuzzing rxe, different approaches, evaluation.
%%
%% Maksyms input:  I would suggest to prioritise the sections where you have more data
%% given that you try diff approaches, sllight different structure of section,
%% instead of normal design, impl and evauluation
%% user level fuzzing, network level fuzzing, evaluation
%%


%% NEW LAYOUT OF INTRO
%%
%% RDMA, what is RDMA. Security issues
%%
%% Role of SoftRoCE
%%
%% Fuzzing


Remote Direct Memory Access (RDMA) enables remote access of data between different machines across a network\cite{rothenbergerReDMArkBypassingRDMA2021}.
This technology is increasingly being deployed in High Performance Computing (HPC) and Cloud settings because
it offers significant performance improvements over traditional networks.

Different works have exposed inherent flaws in the design of RDMA technologies. These flaws present
vulnerabilities that lead to Denial of Service (DoS) attacks, unauthorized memory access, impersonation
attacks through packet injection and performance degradation through resource
exhaustion~\cite{rothenbergerReDMArkBypassingRDMA2021}\cite{tsaiDoubleEdgedSwordSecurity2019}.

This work focuses on exploring errors in current implementations of RDMA, specifically under Linux.
To achieve this goal, we use a software testing technique called fuzzing.

Fuzzing tools or Fuzzers have discovered numerous bugs in the past, partly because
Fuzzers do not necessarily require knowledge of the system under test to do their job. However, to achieve better results,
The Fuzzer can use code coverage information or knowledge about the source code of the target program.

We evaluate different attack surfaces, like the system call interface inside Linux and the
communication channel that RDMA uses. They provide distinct entry points from which
we can fuzz the implementation of the RDMA infrastructure inside Linux.

When evaluating the system call interface, we use a popular system call based Fuzzer named Syzkaller.
Syzkaller generates programs consisting of multiple system calls and runs them in several virtual machines. The
Fuzzer Syzkaller also provides the option to specify the system calls in more detail through a language called Syzlang,
allowing us to target the RDMA subsystem of the Linux Kernel.

For targetting the RDMA communication channel, we design a proxy architecture and carry out the implementation
of a Proxy Fuzzer that allows us to target the device drivers, which are in charge of processing the packet headers
defined by the network layer protocol.

During the fuzzing sessions, both Syzkaller and the Proxy Fuzzer were able to trigger
memory errors and warnings in the RDMA implementation inside current versions of the Linux Kernel.
