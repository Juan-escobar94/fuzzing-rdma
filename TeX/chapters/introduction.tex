\section{Introduction}


% Computer Systems have become indispensable in our everyday life. Examples vary 
% from A to B. As they play important roles in our lives, we expect them to be 
% stable

Remote Direct Memory Access (RDMA) networks offer significant performance benefits by exposing the (network) device directly
to the user application. On one hand RDMA-networks remove the operating system from the critical path of 
communication. But on the other hand, the interface from the application to the operating system and from 
the application to the device becomes significantly more complicated. In comparison to traditional sockets-based API, 
such design increases the attack surface of the OS and the underlying network infrastructure.

With the increasing adoption of RDMA-networks in cloud and HPC settings, the attacker models consider user applications 
to be potentially malicious. Therefore, the interface between the application and the RDMA-network must be secure.
The existing works have already shown fundamental problems in state-of-the art RDMA network architectures, but 
have not yet studied the safety of the API itself. The goal of this work is to characterise the attack surface 
created by the RDMA communication API.

\subsection{Goal}

The goal of this project is to characterise possible attack vectors coming through the RDMA API. Furthermore, the 
project must propose measures to harden the attack surface. A promising approach for hardening RDMA API is to apply 
fuzzing techniques aimed to find potential vulnerabilities in existing kernel-level RDMA-infrastructure. If such 
direction is chosen, the student shall design a fuzzing infrastructure targeting RDMA devices or RDMA device 
drivers. As such, the fuzzing infrastructure may rely on specifics of RDMA-network architectures to make bug-finding 
more efficient.