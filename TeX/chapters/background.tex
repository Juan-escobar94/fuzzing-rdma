\section{background}



\subsection{RDMA Technology}


\subsection{The Linux Kernel}

Talk about the linux kernel being expandable by modules, which operate on
so called `kernel space` or privileged mode, this makes them in turn mandatory
to be well tested, because  a malfunctioning module in this case a device 
driver a potential target inside the system.

\subsection{Fuzzing}

Software testing is a tool used used in software production pipelines, its main goal is to ensure proper program functioning. One common way to test programs is 
positive testing, in which these are tested against expected inputs under well defined scenarious (test cases), to produce the desired results.
Fuzzing is a technique used for negative testing, as opposite to the one just mentioned, in which the program is put under malformed or non-expected input. this
has lead to the discovery of many bugs in the recent years (TODO:\@ cite OSS website from lecture)


\subsubsection{Types of Fuzzers}
 
Fuzzers can be categorized into 3 main types\cite{fetzer20}:

\begin{itemize}
    \item Blackbox: no knowledge of target program internals, therefore they have no means of measuring coverage and can only provide input to the target.
    \item Whitebox: these fuzzers rely on the knowledge of the source code of the target application, making them more efficient ( TODO: symbolic execution tools, model checkers that improve coverage, computed coverage guides the generator)
    \item Graybox: a combination of both blackbox and whitebox types.
\end{itemize}




\subsubsection{Coverage and Depth}

\subsubsection{Fuzzer components}

\paragraph{Generator}

\paragraph{Delivery method}

\paragraph{Monitoring Framework}
