\section{Background}

This section reviews concepts


\subsection{RDMA}

Initially, RDMA was widely available with the InfiniBand (IB) architecture,
which defines network protocols over the link, network and transport layer to move
RDMA packets across the network~\cite{}. The IB network stack requires however
specific hardware that is aware of this protocols.

Later came RDMA over Converged Ethernet (RoCE). This specification
made RDMA accessible to the ubiquitous Ethernet technology and is available
in two versions. RoCEv1 replaces the link layer protocol for the ethernet protocol,
thus enabling the use of regular switches for RDMA services. On top of the modifications
made by RoCEv1, RoCEv2 replaces the network layer protocol of IB for the IP protocol. This
allows routers to transmit RDMA traffic, therefore making RDMA traffic routable under
common hardware.

Machines communicating through RDMA require however a special kind of Network Interface Card (NIC),


A software implementation of RoCE, SoftRoCE


Because SoftRoCE does not require any kind of specialized hardware it plays a major role
in this work.



---OLD PART---

In the setting of modern data centers and HPC clusters, there have been numerous research efforts
to enhance performance. Data Analytics applications that rely on machine and deep learning techniques demand
tremendous workloads with their need for massive computing performance, and have led to the wide adoption of graphics
processing units (GPU) accelerator and multicore (CPU) technologies, which are pushing current datacenter interconnect
and memory architectures to their limit\cite{shalfHPCInterconnectsEnd2019}.

Furthermore, communication using conventional ethernet fabric is a severe inhibitor to efficient resource sharing.
For emerging compute intensive and resource hungry applications, substantial increases in bandwidth are a necessity\cite{shalfHPCInterconnectsEnd2019}.

A shift towards high speed network fabrics has been proven useful,
because it helps alleviaite the latencies that may arise from node to node communication.
The usage of InfiniBand fabrics over Ethernet fabrics is common in such setting.

The amount of time that a processor is engaged in the transmition or reception of a message, also known as overhead,
can also play a major role in performance, since the processor cannot perform other operations.
This is especially true for applications that are heavy on read/write operations, as they can
be slowed down significantly by high overheads\footnote{Overhead}\cite{martin97}.

To tackle this problems, a common solution applied in industry is the usage of InfiniBand, which
is a network architecture which supports RDMA.
RDMA works by letting the Network Interface Card (NIC) and the User Application (more specifically, its memory)
directly communicate with each other, therefore bypassing the Operating System (which is known for
causing high overhead through context switches to kernel space) and thus offloading the CPU. This practice has been
gaining traction in the HPC environment, aiding multiple machines to mitigate bottlenecks and efficiently act like a
single large system.

Taking advantage of the InfiniBand architecture requires specialized hardware, as the InfiniBand network architecture defines
its own protocols for each network layer. Moreover, special NICs, which go by the name of Channel Adapters (CA) in the InfiniBand
architecture are also required. Specifically, A CA is a programmable DMA engine which special protection features that allow
DMA operations to be initated locally and remotely\cite{infinibandvol107}.



%%  take again the main argument talking about softroce and its importance
%%  with possibly more details.
%%
%%  introduce the linux kernel, and talk about relevance of
%%  understanding the compiling process and specific options for this
%%  work.
%%
%%  directly afterwards, talk about device drivers and previous approaches
%%  made to pursue bug free software in this domain, go into fuzzing
%%
%%  explain fuzzing in depth, with specific regard to dev drivers and or
%%  SoftRoCE, talk about well known fuzzers and their limitations in
%%  respect to device drivers / rdma.
%%
%%  talk specifically about projects that use fuzzing to test
%%  device drivers, what have they achieved, what are their limitations
%%
%%  talk about my approaches and what have I achieved

\subsection{The Linux Kernel}


As the SoftRoCE driver operates as a subsystem in the Linux Kernel, it is pertinent to briefly
discuss the implications encompassed by this.

Similar to most Unix kernels, the Linux Kernel is of monotithic nature. All kernel layers
are integrated into a single program, all of which run under the kernel's privileged execution
mode. As a consquence of this, and since security checks are only enforced by the kernel code,
if there are secutiry vulnerabilities inside the kernel, then the whole system has vulnerabilities.

One main feature of the Linux Kernel is its ability to extend its functionality during runtime using
modules. Modules are pieces of code that can be dynamically linked or unlinked at runtime,
adding functionality to the kernel\cite{korbethLinuxDeviceDrivers2005}.

During the process of kernel configuration, device drivers required for InfiniBand support can
be compiled as built-in or as modules (Along SoftRoCE, userpace MAD support,
userspace access (verbs), among others, are required). This should make no difference for our assessment
about bugs and vulnerabilities for the reasons explained above.


\subsection{Fuzzing}

In a nutshell, fuzzing is an approach to software testing where the system being tested is bombarded with test cases generated with another program. The system is then monitored for any flaws exposed by the processing of
this input\cite{mcnallyFuzzingStateArt2012}.

Generally, this test cases consist of increasingly random modified otherwise valid inputs. Although modern
fuzzers do not rely solely on randomness for test case generation, the simplicity of this idea yielded high
effectiveness: fuzzing is still a widely applied technique between both security and quality assurance experts.
It has discovered a considerable amount of bugs over the years.


%% Software testing is a tool used in software production pipelines, its main goal is to ensure proper program functioning. One common way to test programs is
%% positive testing, in which these are tested against expected inputs under well defined scenarious (test cases), to produce the desired results.
%% Fuzzing is a technique used for negative testing, as opposite to the one just mentioned, in which the program is put under malformed or non-expected input. this
%% has lead to the discovery of many bugs in the recent years (TODO:\@ cite OSS website from lecture)

\subsubsection{Types of Fuzzers}

A program that generates input for another program to process, delivers the input and monitors the program
is called a fuzzer. Fuzzers consist of different components, which will be discribe later.

One important aspect for categorizing fuzzers is code coverage, measuring code coverage is usually implemented
with help of compiler instrumentation techniques. In such case, code coverage is measured by the number of
basic blocks reached by program execution given a \textbf{specific input}. Efficient fuzzers are capable of producing inputs that reach high coverage.

Fuzzers can be categorized into 3 main types:

\begin{itemize}
    \item Blackbox Fuzzer: No knowledge of target program internals, the testing process is limited to observing the target's input and output behaviour. Consequently, they have no means of measuring coverage and can only provide input to the target.
    \item Whitebox Fuzzer: These fuzzers can take advantage of the source code of the target application, not only allowing them better insights into how the system works, but also to benefit from code coverage measurements.
      %% these fuzzers rely on the knowledge of the source code of the target application, making them more efficient ( TODO: symbolic execution tools, model checkers that improve coverage, computed coverage guides the generator)
    \item Graybox Fuzzer: This applies to fuzzers that do not fall into any of the previous categories. Source code of the target may not be available, but other means of inspection such as reverse engineering, static analysis, or executing inside a debugger are the usual workhorses of graybox fuzzers\cite{mcnallyFuzzingStateArt2012}.
\end{itemize}

%% Test case generation based on coverage is described by the algorithm in Algorithm~\ref{alg:genalg}.

%% \RestyleAlgo{boxruled}
%% \SetAlgoVlined
%% \begin{algorithm}[h!]
%%   \caption{Algorithm for Coverage-Based Data Generation}\label{alg:genalg}\LinesNumberedHidden\DontPrintSemicolon
%%   \KwIn{A set of Inputs $I$}
%%   \KwOut{An Input Corpus $C$, such that $I \subseteq C$}
%%   $C \gets I$\;
%%   $coverageOld \gets 0$\;
%%   \While{true}{
%%     $newInput \gets generateInput()$\;
%%     $coverageNew \gets runTargetProgram(newInput)$\;

%%     \If{$coverageOld < coverageNew$} {
%%       $C \gets C \cup \{newInput\} $
%%     }
%%     $coverageOld \gets coverageNew$
%%   }
%%   \Return{C}
%% \end{algorithm}

%% %% TODO: highlight importance with an if cascade program.
%% To highlight the importance of using coverage information to generate new test inputs, consider the program in
%% Listing~\ref{lst:ifcascade}. Reaching the potential out-of-bounds access would require a blind fuzzer (i.e. without code coverage
%% information) to go over ${\frac{1}{2}}^{40}$test cases\footnote{${\frac{1}{256}}^{5}$ test cases, as the char type is 8 bits wide.}.

%% \begin{lstlisting}[caption={If-Cascade Program}, label={lst:ifcascade},  style=CStyle, float, floatplacement=h!]
%% char input[6];

%% void read_input() {
%%   fgets(input, 6, stdin);

%%   if (input[0] == 'o') {
%%     if (input[1] == 's') {
%%       if (input[2] == 't') {
%%         if (input[3] == 'u') {
%%           if (input[4] == 'd') {
%%             crash_program();
%%           }
%%         }
%%       }
%%     }
%%   }
%%   operate_normally();
%% }

%% void crash_program() {
%%   int index = random() % 10;
%%   /* potential out-of-bounds access */
%%   input[index] = 'S';
%% }


%% \end{lstlisting}



Along with coverage, depth is also an important metric when evaluating fuzzing efficiency. A program
might reject a test case if, for example, the computed checksum is not equal to the checksum encountered
in a fabricated packet. A later step further might reject the input because a header section does not conform
to a specific standart. Reaching further stages past  checks inside a program corresponds to reaching greater
depths, which is obviously benefitial for succesful fuzzing.

\subsubsection{Fuzzer Components}

Modern fuzzers are composed of the following components:

\paragraph{Fuzz Generator:}

A fuzz generator is in charge of creating test inputs to run on the system under test. Different
approaches serve this purpose\cite{mcnallyFuzzingStateArt2012}:

\begin{itemize}\label{ss:fuzzer-components}
    \item Mutative fuzzer: the idea is to start with a sample input or a set of inputs, and \
    modify a part of it. This set of inputs is commonly refered to as the \textbf{input corpus}; the mutative approach is useful for inputs that must conform to defined structures, coverage is not
    always good. A positive aspect of this type of generators is that they do not require protocol knowledge.
    \item Generative fuzzer: generate inputs from scratch. This can yield higher coverages, at the cost of requiring
    knowledge of protocols. It can be based on templates or grammars.
\end{itemize}

These two main categories can be further described by the techniques used to generate their data:

\begin{itemize}
    \item Oblivious: generates or mutates data randomly, this brings of course poor coverage.
    \item Template based: provided an input template, the fuzzer will only modify/generate specific parts \
    allowed by the template.
    \item Block based: represent data as nested data blocks of varying types as opposed to string sequences.
    \item Grammar based: make use of a previously given grammar, to generate input.
    \item Protocol fuzzer: knows about concepts such as replies and responses, and in which \
    sequence to reply responses in order to test certain functionality.
\end{itemize}

\paragraph{Coverage guided fuzzing:}

Test case generation based on code coverage is an important aspect for efficient Fuzzers.
The Fuzzer generates a new input based on the input corpus it already has. After
running the target program with this new input, it will collect the code coverage
information and compare it to the previous coverage. If the code coverage is higher after
running the target with this newly generated input, the Fuzzer adds the input to its corpus of inputs.
Algorithm~\ref{alg:genalg} describes this procedure.

\RestyleAlgo{boxruled}
\SetAlgoVlined
\begin{algorithm}[h]
  \caption{Algorithm for Coverage-Based Data Generation}\label{alg:genalg}\LinesNumberedHidden\DontPrintSemicolon
  \KwIn{A set of Inputs $I$}
  \KwOut{An Input Corpus $C$, such that $I \subseteq C$}
  $C \gets I$\;
  $coverageOld \gets 0$\;
  \While{true}{
    $newInput \gets generateInput()$\;
    $coverageNew \gets runTargetProgram(newInput)$\;

    \If{$coverageOld < coverageNew$} {
      $C \gets C \cup \{newInput\} $
    }
    $coverageOld \gets coverageNew$
  }
  \Return{C}
\end{algorithm}

%% TODO: highlight importance with an if cascade program.
To highlight the importance of using code coverage information to generate new test inputs, consider the program in
Listing~\ref{lst:ifcascade}. Reaching the potential out-of-bounds access would require a blind fuzzer (i.e. without code coverage
information) to go over ${\frac{1}{2}}^{40}$test cases\footnote{${\frac{1}{256}}^{5}$ test cases, as the char type is 8 bits wide.}.

\begin{lstlisting}[caption={If-Cascade Program}, label={lst:ifcascade},  style=CStyle, float, floatplacement=H]
char input[6];

void read_input() {
  fgets(input, 6, stdin);

  if (input[0] == 'o') {
    if (input[1] == 's') {
      if (input[2] == 't') {
        if (input[3] == 'u') {
          if (input[4] == 'd') {
            crash_program();
          }
        }
      }
    }
  }
  operate_normally();
}

void crash_program() {
  int index = random() % 10;
  /* potential out-of-bounds access */
  input[index] = 'S';
}
\end{lstlisting}

If a Fuzzer leverages code coverage information to generate new inputs, the number of input cases it needs to generate
to discover this deeply nested crashes is substantially reduced. % afl jpeg out of thin air


\paragraph{Delivery Mechanisms}

A delivery mechanism is a fuzzer component which is in charge of presenting input (from the generator)
to the system under test. Normally, as finding abnormal functioning during regular operation of the
target program is aimed at, Fuzzers use delivery mechanisms simulate those used by the system under test\cite{mcnallyFuzzingStateArt2012}.

Delivery mechanisms are usually one of:

\begin{itemize}
    \item Files
    \item Environment variables
    \item Invocation parameters (command-line or API)
    \item Network transmissions
    \item Operating system events (including mouse and keyboard events)
    \item Operating system resources
    \item Direct memory injection (risk to corrupt state and abort program)
\end{itemize}

% (TODO: expand more on this based on which one is chosen)

\paragraph{Monitoring Frameworks}

Detecting that the system under test has crashed as a result of some input is essential to the fuzzing process.
There are two broad classes of monitoring frameworks\cite{mcnallyFuzzingStateArt2012}:

\begin{itemize}
    \item Local monitoring: the fuzzer is installed in the same system as the target. It may look for additional output \
    from the target (such as core dump files) in order to assert that the target has encountered errors.
    \item Remote monitoring: more limited that their counterpart by nature, remote monitowring frameworks recognize \
    a failure by looking for network interruptions.
\end{itemize}

%% IMPORTANT: Fuzzing linux kernel, explain compile options relevant for fuzzing
\subsection{Fuzzing the Linux Kernel}\label{ss:fuzzingkernel}
%% https://youtu.be/YwX4UyXnhz0?t=452
%%

As opposed to userspace fuzzing, fuzzing the Linux Kernel
presents some challenges and limitations. As an example, in single threaded
userspace applications, coverage is a function of the input, it has a
deterministic behaviour; collecting coverage for the kernel however, is
very different;  the kernel contains a high degree
non-deterministic behavior. In \cite{okechInvestigatingExecutionPath2013}, a
simple test program consisting of an open and a read system call revealed
559 distinct execution paths, which were mainly attributed to interrupts and context switches.
% without interrupt and switches 33 unique paths

Finding races is another desirable outcome in this scenario. This is
challenging because this kind of bugs depend on exact
execution interleavings of threads to be triggered, they are consequently hardly reproduceable if the testing method does not account for deterministic
thread interleavings.

Despite this challenges, fuzzing has proven to be excelent approach to testing the
kernel's code base: the system call fuzzer Syzkaller alone has discovered
almost  4000 bugs since the time it has been operational as a continous
automated fuzzer\cite{Syzbot}.

The following configuration options are relevant when fuzzing the kernel:

\begin{itemize}
\item The Kernel Address Sanitizer (KASAN): Using compiler instrumentation, validity checks are inserted to memory accesses, thus revealing memory safety errors such as \
  out-of-bound access and use-after-free. For better error reports containing stack traces, one can also enable the option CONFIG\_STACKTRACE along KASAN\cite{KernelAddressSanitizer}.
\item Kcov: kcov exposes coverage information for the kernel in a form suitable for coverage-guided fuzzing. Coverage data is exposed via the kcov debugfs file, located under /sys/kernel/debug/kcov. Kcov \
  aims not to collect as much coverage as possible, but stable coverage (function of the input), by disabling coverage collection from interrupts and from non-deterministic parts of the kernel\cite{KcovCodeCoveragea}
\end{itemize}

%% TODO panic on warn set.

%%
%% despite of this difficulties, talk about limitations of compiler based tools: they cannot discover many errors past more obvious ones.

%%
%%
%% TODO: remove or rewrite this section, it sucks.
%% \subsection{Fuzzers REMOVE OR REWRITE}

%% There are well over dozens of open source fuzzing tools and libraries \cite{awesome-fuzzing}, but the purpose of this work, the following
%% are well worth mentioning:
%% % TODO: in each section, why not chosen.


%% \paragraph{AFL}

%% The American Fuzzy Lop fuzzer is a brute force fuzzer that uses instrumentation guided coverage to
%% efficiently fuzz the system under test \cite{afl}. AFL can operate as a white-box fuzzer (source code is availabljkke)
%% to achieve best results, this requires compiling the source code with a drop in replacement for gcc/clang
%% called afl-gcc.


%% \paragraph{Libfuzzer}

%% Libfuzzer is a coverage guided evolutionary fuzzing engine, it was designed with the intent to fuzz
%% libraries, feeding fuzzed inputs to an entry point (target function). In order to achieve this Libfuzzer
%% is linked with the library under test. It can be combined with AFL for example, this would require both processes to be
%% periodically restarted in order to 'use' each others findings \cite{libfuzzer}.


%% \paragraph{kAFL}

%% % problem: requires intel PT
%% % TODO: clean up this section.

%% As AFL is limited to userspace and lacks support for kernel space\cite{kafl}, kAFL was designed with overcoming
%% the obstacles that are inherent when fuzzing kernels: kernel crashes (panics) usually make the host operating system
%% unavailable, collecting coverage from a broken system is neither possible, nor perhaps desired.

%% The inherent non-determinism in kernel level programs: kernel threads, statefulness and similar mechanisms. (libfuzzer)
%% this makes fuzzing kernel code challenging.

%% coverage information is provided by Intel's Processor Trace (PT) technology. Logic used to perform
%% mutations and scheduling closely resembles the one used in AFL\cite{kafl}

%% \paragraph{Syzkaller}

%% Syzkaller is a fuzzing framework designed to fuzz kernels at the system call level. It operates on multiple
%% virtual machines and collects coverage information using KCOV. Syzkaller also comes with a systemcall specification
%% language (syzlang), designed to aid in targetting specific subsystems inside kernels \cite{syzkaller}.

\subsection{RDMA Applications}

%% was released in 2002, bit historical, but I want to mention this
%% to say what libibverbs are
The ``RDMA Verbs Specification' internet draft describes the abstract
interface for RDMA-aware NICs and provides a semantic definition of this
interface. This semantic definition is simply called Verbs\cite{hillandRDMAProtocolVerbs}.

Libibverbs provides an implementation of the Verbs interface\footnote{Hosted at \url{https://github.com/linux-rdma/rdma-core}}. User space
applications can consume the API of libibverbs to interact with
the kernel modules for control path operations and to interact with the drivers asociated
to RDMA-aware devices.

The libibverbs API handles the control path for creation, modification, querying and destruction of resources used by RDMA applications.

For the control path, the library uses system calls that are handled by file operations on device nodes
registered by the uverbs kernel module. The kernel module passes control down to lower-level drivers.
These lower-level drivers can be drivers for real hardware, or drivers for software devices, like SoftRoCE.

In contrast to the control path,  the library implements the data path with calls that are made
directly to the low level drivers of the NIC using a doorbell mechanism: the host CPU %via Programmed IO
writes a short doorbell message directly to the
NIC, indicating there is work to be done\cite{KaliaUsingRDMAEfficiently2014}. This data flow mechanism completely circumvents the
host Kernel, enabling the Kernel bypass\cite{kaliaDesignGuidelinesHigh2016}.

The Queue Pair one of the most important resources managed by RDMA applications, it represents an endpoint
in a communication channel. There are different transport modes that can be selected when establishing a Queue Pair\cite{rdmamanual}:

\begin{itemize}
  \item Reliable Connection (RC): A Queue Pair is associated with only one other Queue Pair. Similar to TCP connections, packets are acknowledged and guaranteed to arrive in order.
  \item Unreliable Connection (UC): A Queue Pair is associated with only one other Queue Pair, packet delivery is not guaranteed.
  \item Unreliable Datagram (UD): A Queue Pair may receive or transmit single packets from/to any other UD queue pair.
\end{itemize}

After being created, a Queue Pair is still not ready to receive or send data through a channel; the Queue Pair must be first
transitioned through several states, shown in Figure \ref{fig:qpstatemachine}.

\begin{figure}[h!]
  \centering
  \includegraphics[height=8cm, keepaspectratio]{qpstatemachine}
  \caption[Queue Pair state machine]{The Queue Pair state machine, slightly adapted from \cite{QPStateMachine2012}.}
  \label{fig:qpstatemachine}
\end{figure}

%% A WQ is either a SEND work queue or a RECEIVE work queue
%% QPs have both queues associated to them

The Queue Pair consists of two Work Queues, a Send Queue and a Receive
Queue. The Work Queues handle sending and receiving messages.
An application can post Work Requests in the context of a Queue Pair, in
doing so, the application asks the RDMA device to handle the task.
A Work Request can be polled for completion, this operation will yield
the status of the Work Request that was posted. Depending on the transport
mode and the Work Request, a successful status can depend on the receiver
side to answer with an acknowledgment packet.


Directly after creation, a Queue Pair is in a Reset state. In the Reset state,
a Queue Pair cannot post any Work Requests to any
of its Work Queues, and any packets it receives will be silently dropped.
A Queue Pair can transition from one state to another when the program calls ibv\_modify\_qp() ---
transitions marked with blue in Figure \ref{fig:qpstatemachine} originate from this function call.
Work Requests may be posted to the Receive Queue once the Queue Pair has transitioned to a Ready To Receive (RTR) state,
similarly Work Requests may be posted to the Send Queue after reaching the Ready To Send (RTS) state. The Queue Pair will silently drop
packets until it has at least reached the RTR state.

The Send Queue Error (SQE) state transition happens automatically
after a Work Request in the Send Queue ends with an error\cite{QPStateMachine2012}. The Send Queue Drain (SQD) is a special state
that won't process further Work Requests posted to the Send Queue\cite{QPStateMachine2012}. Both the SQD and SQE state handle incoming packets\cite{barakVerbsProgrammingTutorial2014}.

Errors ocurring during the ibv\_modify\_qp() call will bring the Queue Pair
to the Error state, Work Request errors may transition the Queue Pair to this state as well\cite{QPStateMachine2012}.


\subsubsection{Transport Layer: The Base Transport Header} % OR: role of the device driver before

One of the tasks of low level drivers for RDMA devices is to process
packet headers of the transport layer protocol. This task roughly consists
of assigning a queue pair to the packet, performing requested memory access operations like
read or write and performing sanity checks.

These transport layer headers are called
base transport headers, and contain information describing the
requested operation and the  destination queue pair.

There are also header extensions for the base header, but their presence is dictated
by the operation code. We will focus in this section on base transport header,
because they are common to all RDMA traffic.

Base transport headers are 96 bits long, Table
\ref{tab:bthfields} shows the corresponding field descriptions. The order in which fields
are listed in Table \ref{tab:bthfields} coincide with the layout of the actual header.

\begin{table}[h]
  \begin{tabular}{| m{10em} | m{3em} | m{15em} |}
    \hline
    Field Name &  Field Size (bits) & Description\\
    \hline
    Opcode & 8 & Indicates the packet type, also specifies which extension headers follow the base transport header.\\
    \hline
    Solicited Event & 1 &  The requester sets this bit to 1 to specify that the responder shall invoke the Completion Queue event handler.\\
    \hline
    MigReq & 1 &  Used to communicate Path Migration State, path migration refers to both communication partners agreeing to use a new communication path\\
    \hline
    Pad Count & 2 &  Number of pad bytes added to the headers to align the payload to a 4 byte boundary.\\
    \hline
    Transport Header Version & 4 &  Specifies the Transport version used for the packet, it is set to 0x0.\\
    \hline
    Partition Key & 16 &  Partitioning allows isolation among systems sharing the same fabrics. If using partition, this field is required to match partition key stored at the receiver or else the packet will be discarded\\
    \hline
    Reserved (variant) & 8 &  transmited as 0s, this field is ignored on the receiver side.\\
    \hline
    Destination QP & 24 &  specifies a number to identify the destination queue pair.\\
    \hline
    Acknowledge Request & 1 &  If this bit is set, responder should schedule an acknowledgment packet on the associated queue pair.\\
    \hline
    Reserved & 7 &  transmited as 0s, ignored by receiver. \\
    \hline
    Packet Sequence Number & 24 &  Identifies the position of a packet in a sequence of packets, used e.g. to recognize lost packets.\\
    \hline
  \end{tabular}
  \caption[Base Transport Header Fields]{Base Transport Header Fields, adapted from\cite{infinibandvol107}}
  \label{tab:bthfields}
\end{table}

%% why? explain queue pair state machine, show states requiring a connection
\subsubsection{Challenges posed for Fuzzing}\label{s:ibverbs-challenges}

Reaching full coverage for RDMA Verbs applications is not trivial. In order to perform RDMA operations, establishment of
a connection to the remote host, as well as appropriate permissions need to be set up first. The mechanism for accomplishing
these tasks is the Queue Pair\cite{rdmamanual}. A Queue Pair defined for connected transport modes needs information from
the remote Queue Pair in order to transition to a ready state.

At the time the device driver checks the Base Transport Headers, it will also assert if there is an existing Queue Pair
with the Queue Pair number found in the Destination QP field in the header.
It will also check if this queue pair is in a ready state; if any of these conditions
are not met, the driver will drop the packet.

This means that a fuzzing application should either let 2 applications initialize and reach a ready state,
or lead the target application to think it has connected with another application. As we have seen
in the previous sections, Queue Pairs are modeled by a state machine and they must be carefully transitioned from one
state to another. Since driver code enforces that the Queue Pair must be in a specific state for operations like
handling incoming packets, there might be bugs that are tied to specific states. It is meaningful to test
RDMA applications running in a connected environmnent.

% \subsubsection{Programming with ibverbs}

%% TODO how does it work, basics, state machine
%% specific bug => requires a specific state



%% It is pertinent to evaluate some aspects of linux and the kernel as the nature of
%% the system in which the SoftRoCE driver operates, allows its code to operate
%% under the most privileged level, along the kernel itself.

%% One of the main features of the linux kernel is its ability to extend
%% its functionality during runtime. Code that can be added (or removed) to the kernel
%% while is running are called modules. Each module is made of object code that can
%% be dynamically linked with the insmod or modprobe executables\cite{ldd3}.

%% Security checks in linux are enforced in the kernel code, if there are security vulnerabilities inside
%% the kernel, then the whole system has vulnerabilities.\cite{ldd3}

% Talk about the linux kernel being expandable by modules, which operate on
% so called `kernel space` or privileged mode, this makes them in turn mandatory
% to be well tested, because a malfunctioning module in this case a device 
% driver a potential target inside the system.

% TODO: move this to methodology
%% When compiling the linux kernel, we want to enable support for the InfiniBand network stack, virtualization and
%% also builtin functionality that will come to aid during the fuzzing process, the following are worth
%% mentioning:

%% \begin{itemize}
%%     \item Kcov: Code coverage for fuzzing. From \cite{kerneldocs-kcov} Kvoc exposes kernel code coverage information in a form suitable\
%%     for coverage-guided fuzzing (randomized testing). Coverage data of a running kernel is exported via\ Background
%%     a debugfs file called kcov. % TODO:
%%     % NOTE ON KCOV profiling data will only become accesibleonce debugfs has been mounted mount -t debugfs none /sys/kernel/debug
%%     \item KASAN: The Kernel Address Sanitizer is a dynamic memory error detector. It provides\
%%     information for finding use-after-free and out-of-bound bugs, it uses compiler generated\
%%     instrumentation code to check every memory access \cite{kerneldocs-kasan} % TODO:
%%     \item Fault Injection: interest to test error handling paths of IO operations, malloc, futex calls, etc: reason => a big portion of kernel code is concerned with error handling operations. => related work
%%
%% \end{itemize}

%% TODO: mention UBSAN and KCSAN


%% \subsection{RDMA with Infiniband}

%% RoCE is a standard for RDMA over Ethernet, which allows leveraging RDMA semantics over an Ethernet network without needing
%% to resort to TCP transport.
%% % According to \cite{rdmamanual} RoCE provided the most efficient low lattency Ethernet solution to the day of (their?) writing (in 2015).
%% % TODO: maksym => just say that RoCE is generally cheaper because it does not require specialized switches.
%% \paragraph{Rxe: Soft-RoCE}

%% Soft-RoCE is a software implementation of the RDMA transport (over Ethernet), it has been developed as a github
%% community project, with help from IBM, Mellanox and other companies.This software driver provides us an
%% uncomplicated manner to test RDMA technologies without needing to use real hardware, it provides a complete RDMA
%% stack implementation over any NIC \cite{mellanox-community}.
