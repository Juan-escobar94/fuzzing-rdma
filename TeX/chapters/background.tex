\section{Background}

\subsection{The Linux Kernel}

% Linux distributions

One of the main features of the linux kernel is its ability to extend 
its functionality at runtime. Pieces of code that can be added (or removed) to the kernel 
while it is up and running are called moules. Each module is made of object code that can 
be dynamically linked with the insmod or modprobe executables.\cite{ldd3}, 

Security checks in linux are enforced in the kernel code, if there are security holes inside
the kernel, then the whole system has security holes \cite{ldd3}

% Talk about the linux kernel being expandable by modules, which operate on
% so called `kernel space` or privileged mode, this makes them in turn mandatory
% to be well tested, because a malfunctioning module in this case a device 
% driver a potential target inside the system.


% TODO: explain that the kernel can be built with different configurations 

In order to get a a custom kernel configuration that suits the needs of this work,
the latest stable release (5.10.11 at the time of writing) was downloaded from kernel.org,
the downloaded bundle contains the entire source code (called source tree) for the linux kernel.

With this source tree, one can create a linux kernel, tailored down to a specific use case. In 
this scenario we want to enable support for the InfiniBand network stack, virtualization and 
also builtin functionality that will come to aid during the fuzzing process, the following are worth
mentioning:

\begin{itemize}
    \item KCOV: code coverage % TODO:
    \item KASAN: kernel adress sanitizer ... % TODO:
\end{itemize}

There are some requirements to build the kernel, the list of programs needed for compiling the kernel are
listed under \cite{minreqskernel}.

inside the kernel source tree main directory, running make causes the kernel build system to use the configuration 
described inside a file named .config to build a kernel and all modules needed to support that configuration 
\cite{kroah-hartman06}.

\subsection{RDMA with Infiniband}

% TODO: is this necesary? are we using specifically InfiniBand technology? => modules are listed under /drivers/infiniband, so Ill guess yes

InfiniBand (IB) is a network architecture which provides very efficient and scalable server and storage interconnect
technology, which is capable of supporting Remote Direct Memory Access (RDMA). the IB technology enables data 
transmission between servers without needing CPU intervention,reducing in this way communication overhead, providing 
speeds going up to 56Gb/s per port\cite{rdmamanual}.

% TODO: Maksym: Should I talk about virtual protocol interface (VPI)?

\paragraph{RDMA over Converged Ethernet (RoCE)}

RoCE is a standard for RDMA over Ethernet, which allows leveraging RDMA semantics over an Ethernet network without needing
to resort to TCP transport. According to \cite{rdmamanual} RoCE provided the most efficient low lattency Ethernet
solution to the day of (their?) writing (in 2015).

\paragraph{Rxe: Soft-RoCE}

Soft-RoCE (short rxe) is a software implementation of the RDMA transport (over Ethernet), it has been developed as a github 
community project, with help from IBM, Mellanox and other companies.It has been part of the kernel since version 4.8.
This software driver provides us an uncomplicated manner to test RDMA technologies without needing to use real hardware, it provides a complete 
RDMA stack implementation over any NIC \cite{mellanox-community}.

running \\ \$ modinfo rdma\_rxe \\ inside the shell, reveals that this module depends on the following modules: ib\_core, ip6\_udp\_tunnel, udp\_tunnel, ib\_uverbs 

all which are as well loaded into the kernel when issuing the command \\ \$ modprobe rdma\_rxe \\
modprobe loads a module into the kernel, while also loading the modules on which it depends on \cite{ldd3}.


% udev is a program that enables Linux to provide a persistent device-naming system
% in the /dev directory \cite{kroah-hartman06}

% To get Soft-RoCE capabilities, you need to install the kernel and user space libraries on both servers:
% We are mainly interested in the rdma\_rxe kernel module, Software RDMA over Ethernet, provides a software implementation of the RoCEv2 Protocol, cite manpages.


\subsection{Fuzzing}

Software testing is a tool used in software production pipelines, its main goal is to ensure proper program functioning. One common way to test programs is 
positive testing, in which these are tested against expected inputs under well defined scenarious (test cases), to produce the desired results.
Fuzzing is a technique used for negative testing, as opposite to the one just mentioned, in which the program is put under malformed or non-expected input. this
has lead to the discovery of many bugs in the recent years (TODO:\@ cite OSS website from lecture)

The term Fuzzing was coined by Miller, as ... \

\subsubsection{Types of Fuzzers}
 
Fuzzers can be categorized into 3 main types\cite{fetzer20}:

\begin{itemize}
    \item Blackbox: no knowledge of target program internals, therefore they have no means of measuring coverage and can only provide input to the target.
    \item Whitebox: these fuzzers rely on the knowledge of the source code of the target application, making them more efficient ( TODO: symbolic execution tools, model checkers that improve coverage, computed coverage guides the generator)
    \item Graybox: a combination of both blackbox and whitebox types.
\end{itemize}


\subsubsection{Coverage and Depth}

\subsubsection{Fuzzer components}

\paragraph{Generator}

\paragraph{Delivery method}

\paragraph{Monitoring Framework}

\subsubsection{Fuzzing Opensource Projects}

\paragraph{AFL}

\paragraph{Libfuzzer}

\paragraph{Syzkaller}

\paragraph{kAFL}