\section{Methodology}

\subsection{Design of a Proxy Fuzzer}

Since many state-of-the-art fuzzers do not directly address some of the challenges
posed by RDMA applications (refer back to \ref{s:ibverbs-challenges}), I propose a fuzzer that plays the role
of a proxy between two machines comunicating using RDMA. The network architecture
allows the fuzzer not only to monitor traffic between the machines, but also to modify it.
Unlike system call based fuzzers, this approach can easily test code paths which require
fully initialized and connected applications, in the context of % ibverbs.

%% This presents an advantage over system call based fuzzers, in such a way that
%% it can easily trigger code paths that require fully initialized and connected applications,
%% in the context of the Infiniband Verbs semantics.

As the goal of the fuzzer is to trigger bugs related to the RDMA network stack,
only the modification of headers that are pertinent to it is desirable;
the Infiniband Base Transport Headers, sitting at layer 4,
define fields related to RDMA applications. Fuzzing any other headers such as
ethernet or IP headers would be detrimental to efficiency, as the modified packets may
not even reach the target application. Fuzzing payload contents is also undoubtedly inefficient,
because it does not concern RDMA device drivers.


It is relevant to mention that even though this design was conceived with
SoftRoCE at hand, it can be applied to anything that processes Infiniband Base
Transport Headers, even real NICs. This is true because processing of Infiniband
Base Transport Headers is implemented at the hardware level.


%% TODO: (backlink to background, remember to explain about state of Queue Pairs).

%% TODO: add a section with anatomy of a base transport header

At the first stage of the conceptual design this work first evaluates the impact of fuzzing
traffic in only one direction, but it will expand onto the possibility of attacking the
communication in both directions. Figure \ref{fig:fuzzerconcept} illustrates the underlying
concept for the described design.

\begin{figure}[h]
  \centering
  \includegraphics[width=\linewidth]{proxyfuzzerconcept}
  \caption[Proxy fuzzer concept]{basic concept of a proxy fuzzer, red arrows denote fuzzed traffic.}
  \label{fig:fuzzerconcept}
\end{figure}

\subsection{Implementation} % maybe this is still part of design

With the help of the machine emulator / virtualizer QEMU, ///I can create a pair of
virtual machines running the custom kernel on a x86\_64 architecture.
Using debootstrap, ///one can easily install a Debian system on
the virtual machines, and specify all of the required packages
for ///my test environment, including those necessary to compile and run applications
using the verbs libraries.

We///// want to be able to see traffic, % tap interfaces offer a solution

By using TUN/TAP interfaces, the fuzzer can read all network traffic that goes
through the TUN/TAP interface inside a user space program. As stated in
the Kernel documentation, these interfaces can be seen as a simple point-to-point
or ethernet device, which send and receive packets to and from a user space program.

I///// can connect the network interface cards
inside the virtual machines to a virtual interface visible to the host environment.

% briefly explain tap interfaces

Creating a TAP interface is possible by opening the `/dev/net/tun` device file.
With the file descriptor resulting from the previous open system call, we can ask the operating system
to create a TAP interface by issuing an ioctl system call, with the name of the interface we want to
register. Once the program ends, the operating system removes this interface, unless another ioctl call
is issued to make this interface persistent. The creation of the interface and the further ioctl call for persistence
is roughly what the ip tuntap command does\footnote{iproute2: https://git.kernel.org/pub/scm/network/iproute2/iproute2.git/ }.

% TODO: code snippet for creating such devices


Unfortunately, we cannot reopen these interfaces in the context of proxy fuzzer process, since the QEMU
process has already opened them; the respective ioctl system call fails and sets the
global ERRNO variable to EBUSY (device or resource busy). To overcome this, we create a pair of
virtual bridges, connect them to the tap interfaces mapped to the ethernet interfaces inside the virtual
machines, and we create additional tap interfaces and also connect them to the bridge.

%% TODO: Picture
%%
\begin{figure}[h]
  \centering
  \includegraphics[width=\linewidth]{proxyfuzzernetworkdetailed}
  \caption[Proxy fuzzer network architecture]{Detailed network architecture for the proxy fuzzer}
\end{figure}

%% TODO: explain fuzzing loop, which fields not to fuzz, removed crc checks

%% TODO: mention other approaches I tried (maybe in discussion)

%% TODO: mention the benchmarks used and the program I devised myself to exchange infiniband traffic.

%% TODO: mention implementation details, like header dumps for reproducibility, number of fuzzing rounds, fuzz after so many IB packets sent, etc.

%% TODO: mention or implement AND mention: monitoring, continuous fuzzing.

%% TODO:

%% TODO: evaluate other fuzzers
