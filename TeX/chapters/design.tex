\section{Design}\label{s:design}

Since many state-of-the-art fuzzers do not directly address some of the challenges
posed by RDMA applications (see Section \ref{s:ibverbs-challenges}), I propose a fuzzer that plays the role
of a proxy between two machines which comunicate using RDMA. The network architecture
allows the fuzzer not only to monitor traffic between the machines, but also to modify it.
Unlike system call based fuzzers, this approach can easily test code paths which require
fully initialized and connected applications, in the context of RDMA Verbs. % ibverbs.

As the goal of the fuzzer is to trigger bugs related to the RDMA network stack,
only the modification of headers that are pertinent to it is desirable;
the Base Transport Header, sitting at layer 4,
defines fields relevant to RDMA applications (see Table \ref{tab:bthfields}). Fuzzing any other headers such as
ethernet or IP headers would be detrimental to efficiency, as other parts of the network stack may not be able
to deliver the packets; the modified packets would
not even reach the target application. Fuzzing payload contents is also undoubtedly inefficient,
because it does not concern RDMA device drivers.

It is relevant to mention that even though this design was conceived with
SoftRoCE at hand, it can be applied to anything that processes Base
Transport Headers, even real NICs. This is true because processing of
Base Transport Headers is implemented at the hardware level. %% maybe cite this?

\begin{figure}[h]
  \centering
  \includegraphics[width=\linewidth]{proxyfuzzerconcept}
  \caption[Proxy fuzzer concept]{Basic concept of a Proxy fuzzer, the red arrow denotes fuzzed traffic.}
  \label{fig:fuzzerconcept}
\end{figure}

Figure \ref{fig:fuzzerconcept} depicts the conceptual architecture for the Proxy Fuzzer.
The traffic can be fuzzed in any direction, including both directions simultaneously.
Nevertheless, as network protocols must conform to an underlying structure and excessive header manipulations usually
just lead to dropped packets, the conceptual
design is mainly concerned with fuzzing in one direction, as shown in figure \ref{fig:fuzzerconcept}.
