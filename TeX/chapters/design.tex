\section{Design}\label{s:design}

%% First motivate your architecture
%% > Why did you chose proxy architecture
%% > Why in one direction
%% > What kind of problem do you try to solve, others dont
%%
%% Then explain how it works
%%
%% then xplain what useful new properties you get
%%
%% do not forget coverage information: say one
%% needs it in principle, but in the implementation
%% I can say I did not do it, because of time constraints.

We want to address specific challenges that arise when fuzzing
RDMA appications. We have seen in Subsection~\ref{s:ibverbs-challenges}
that to test specific parts of the code, the RDMA API requires a connected environment, i.e.
that an application has a Queue Pair that reaches one of the ready states; otherwise,
the code of the device driver in charge of processing RDMA traffic will branch out
to simply drop the traffic and many code paths will remain untested.

\subsection{A Proxy Architecture}

A sensible choice for testing connected applications is to
target the communication channel that these applications use.
A proxy architecture allows a process to sit in the middle of two applications
and read and/or modify the traffic before forwarding such traffic to the original destination.
We will refer to this process in the middle as the proxy fuzzer. Simply put, the proxy fuzzer
takes incoming traffic from an RDMA application and forwards it to another RDMA application.
In doing so, the proxy fuzzer can modify the traffic in such a way that it specifically targets the RDMA stack of a target machine.

\begin{figure}[h]
   \centering
   \includegraphics[width=\linewidth]{proxyfuzzerconcept}
   \caption[Proxy fuzzer concept]{Basic concept of a Proxy fuzzer, the red arrow denotes fuzzed traffic.}\label{fig:fuzzerconcept}
\end{figure}

Figure~\ref{fig:fuzzerconcept} depicts the concept of this architecture. Because we also want to be able to differentiate
easily which packets may cause specific errors, we separate the applications into two different environments by
using two virtual machines. In case of a kernel panic or any other kind of crash, the host environment remains unaffected
and can retain the series of packets that were responsible for triggering the crash.

The fuzzer only modifies the packet headers of RDMA traffic, because it is concerned
with testing the RDMA stack and the RDMA stack only. Thus, the Base Transport Header and Header extensions
are the actual fuzzing targets.  If the proxy fuzzer were to modify IP or Ethernet Headers,
the packets would most likely not reach their destination. Fuzzing other
network protocols is out of the scope of this work. The RDMA payload is completely ignored,
because RDMA device drivers are also not concerned with it. % TODO:\@ As we have already said this work is mainly concerned with fuzzing rdma with softroce as our entry point (or something similar)

Figure~\ref{fig:fuzzerconcept} shows that the traffic is fuzzed only in one direction. The reason
behind this is the nature of network protocols. Traffic for network protocols must conform to an underlying
structure and excessive header manipulations usually just leads to dropped packets. The code coverage information provided by
kcov is collected from the target machine by transmitting coverage information through the network. Code
coverage information is, as we have mentioned, an important aspect of efficient bug discovery.

\subsection{Advantages of this Approach}

By having knowledge of the Protocol, the proxy fuzzer can make conscious modifications to fields in the Base Transport
Header (See Table~\ref{tab:bthfields}), in the hopes of encountering bugs. We refer to conscious modifications here,
because some fields are required to always have specific values (e.g. Transport Header Version), these fields must remain unmodified.

The main advantage is that with this architecture, the proxy fuzzer is able to test code paths
that were previously untested by the tools presented in Subsections~\ref{ss:difuze},~\ref{ss:syzkaller}~and~\ref{ss:fastsyzkaller}.

It is relevant to mention that even though this design was conceived with
SoftRoCE at hand, it can be applied to anything that processes Base
Transport Headers, even real RDMA-enabled NICs. This is true because processing of
Base Transport Headers is implemented at the hardware level.



%% To specifically target the RDMA network stack,
%% We can connect two RDMA applications in different machines and





%% Mention that the goal of this design is to test the driver.
















































%% Since many state-of-the-art fuzzers do not directly address some of the challenges
%% posed by RDMA applications (see Section \ref{s:ibverbs-challenges}), I propose a fuzzer that plays the role
%% of a proxy between two machines which comunicate using RDMA. The network architecture
%% allows the fuzzer not only to monitor traffic between the machines, but also to modify it.
%% Unlike system call based fuzzers, this approach can easily test code paths which require
%% fully initialized and connected applications, in the context of RDMA Verbs. % ibverbs.

%% As the goal of the fuzzer is to trigger bugs related to the RDMA network stack,
%% only the modification of headers that are pertinent to it is desirable;
%% the Base Transport Header, sitting at layer 4,
%% defines fields relevant to RDMA applications (see Table \ref{tab:bthfields}). Fuzzing any other headers such as
%% ethernet or IP headers would be detrimental to efficiency, as other parts of the network stack may not be able
%% to deliver the packets; the modified packets would
%% not even reach the target application. Fuzzing payload contents is also undoubtedly inefficient,
%% because it does not concern RDMA device drivers.

%% It is relevant to mention that even though this design was conceived with
%% SoftRoCE at hand, it can be applied to anything that processes Base
%% Transport Headers, even real NICs. This is true because processing of
%% Base Transport Headers is implemented at the hardware level. %% maybe cite this?

%% \begin{figure}[h]
%%   \centering
%%   \includegraphics[width=\linewidth]{proxyfuzzerconcept}
%%   \caption[Proxy fuzzer concept]{Basic concept of a Proxy fuzzer, the red arrow denotes fuzzed traffic.}
%%   \label{fig:fuzzerconcept}
%% \end{figure}

%% Figure \ref{fig:fuzzerconcept} depicts the conceptual architecture for the Proxy Fuzzer.
%% The traffic can be fuzzed in any direction, including both directions simultaneously.
%% Nevertheless, as network protocols must conform to an underlying structure and excessive header manipulations usually
%% just lead to dropped packets, the conceptual
%% design is mainly concerned with fuzzing in one direction, as shown in figure \ref{fig:fuzzerconcept}.
