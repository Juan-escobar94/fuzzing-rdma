\section{Evaluation}\label{s:evaluation}

This section is dedicated to present the results we gathered
by fuzzing the RDMA stack with different fuzzing tools. We provide
a brief description about the setting in which the RDMA system was fuzzed
and we also provide concise information about the bugs we found. Fixing these
bugs is out of the scope of this work.


%% In this section present the results achieved with different
%% fuzzing tools and explain under which settings we attained this results
%% in each subsection.

\subsection{Syzkaller}

We ran Syzkaller under two different Kernel versions: 5.7.0 and 5.10.11. The latter was
the latest stable version available at the time we evaluated Syzkaller.

We extracted the necessary system calls for specifically targetting the modules
in charge of RDMA in the Linux Kernel. We wrote a simple python script
that reads the Syzlang definitions from ``syz/linux/dev\_infiniband\_rdma.txt'' inside the
Syzkaller source tree. The script uses regular expressions to identify the system call definitions
in the syntax of Syzlang and outputs the trimmed names of all system calls found in the file (without any arguments)
as comma separated values into another file. This was necessary, because we want to restrict Syzkaller
to fuzz RDMA modules in the Linux Kernel. We do this by providing all the previously extracted system calls
as comma separated values to the configuration parameter ``enable\_syscalls'' of Syzkaller.

For each of the Kernels, we let Syzkaller run for a time window of approximately four hours.

For the 5.10.11 Kernel, Syzkaller found one crash and two warnings:
\begin{itemize}
  \item A warning that originates from calling ib\_uverbs\_post\_send. The error report suggests that something goes wrong during page allocation inside the memory manager, % workdir/20b96
  \item A warning emerging from calling ib\_uverbs\_unmarshall\_recv, similar to the previous error report, this warning points to page allocation mechanisms inside the memory manager,
  \item A general protection fault (Null pointer dereference) reported by KASAN, triggered by a call to the function ib\_umem\_get, which is in charge of pinning\footnote{To pin memory is to lock a memory region into RAM:\@ it is guaranteed to be in physical memory, meaning that this memory region will not be swapped\cite{kerriskLinuxProgrammingInterface2010}.} and mapping user space memory for Direct Memory Access.  % workdir/48a33
\end{itemize}

For all of the bugs mentioned above, Syzkaller was able to generate a C reproducer program.

For the 5.7.0 Kernel, Syzkaller found three distinct bugs: the first two coincide with the warnings found in the 5.10.11 Kernel.
The third bug it found during this running session was a warning raised during the execution of the function uverbs\_uobject\_fd\_release.
Syzkaller could not generate a C reproducer for this error and the warning ocurred only once during the entire test session.

\subsection{Proxy Fuzzer}

We ran the proxy fuzzer for a time window of approximately three hours on the 5.7.0 Kernel.
We used different popular RDMA benchmark applications called Performance Tests from the core
Linux RDMA repositories\footnote{Hosted at: https://github.com/linux-rdma/perftest}. These benchmarks
specifically test the bandwidth and the latency of RDMA operations between two machines.
We used tests for RDMA Read, Send, Write and Atomic operations.

%% MAKSYM: MENTION WHY NOT THE OTHER KERNEL => WHERE?

After setting up the network architecture from Section~\ref{s:implementation}, installing the benchmarks in the
virtual machines and letting our Fuzzer run for the three hour time window, the Proxy fuzzer found two distinct bugs:

\begin{itemize}
  \item A warning inside the SoftRoCE driver, emerging from rxe\_completer function inside the rxe\_comp.c file,
  \item A null pointer dereference reported by KASAN, apparently emerging from a call to rxe\_do\_task defined in the file rxe\_task.c, which is also part of the SoftRoCE driver.
\end{itemize}

Different memory errors reported by KASAN ocurred when executing benchmarks
for RDMA Atomic operations, for both bandwidth and latency tests.
Because this errors also ocurr when we turn down the header modifications
of the proxy fuzzer, we cannot attribute the finding of the errors that
were product of running benchmarks for RDMA Atomic operations to the proxy fuzzer.
