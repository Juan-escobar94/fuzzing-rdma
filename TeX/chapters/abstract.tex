
\begin{abstract}
  Nowadays, RDMA networks are being deployed in different environments which rely on high performance.
  The design of the network architecture of RDMA has  concentrated on performance, and previous works have
  proven that this design has inherent design flaws that expose
  various types of security concerns.

  In this paper, we focus on exploring the current implementation of the RDMA transport inside the Linux
  Kernel for implementation errors. To achieve this goal, we apply an efficient software
  testing technique that has discovered numerous bugs in different applications, namely fuzzing. The system call interface of the Linux
  Kernel and the network transmissions necessary for RDMA build the attack surface that we apply fuzzing on. For the system call interface, we
  use a popular system call-based Fuzzer named Syzkaller. For the network
  transmissions, we design a network architecture and implement an RDMA
  packet Fuzzer. Both approaches find memory errors and warnings in the
  RDMA infrastructure of the Linux Kernel.
\end{abstract}
