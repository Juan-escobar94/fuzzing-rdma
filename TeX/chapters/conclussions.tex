\section{Conclusion}

This work set as its goal to search implementation errors in current versions of the RDMA architecture of the Linux Kernel.
To achieve this, an effective software testing technique, namely fuzzing, was applied.
We approached the task at hand through the system call interface of the Kernel and through the network transmissions that RDMA low-level
drivers process. Both approaches were able to trigger warning messages and found memory errors reported by KASAN.\@
We believe that the nature
of the warnings and errors found by the system call-based approach to be more shallow because it does not require the same resource initialization that RDMA applications undergo.  For the network approach,
we design a proxy architecture that allows us to modify the ongoing traffic of two connected applications. Because this approach can test code paths that require proper initialization of resources and connection
establishment, we believe that the warnings and errors found in this way
could not have been easily discovered by the other approach.

After our findings with both Syzkaller and the Proxy Fuzzer, we also conclude that Fuzzing is a
very appropriate approach to test even enterprise-level software. Fuzzing has virtually a minimum requirement
of almost no effort to implement, but the results of the fuzzing process heavily depend on the
amount of time and resources invested.
